\chapter{Einleitung}

\begin{fquote}[Mark Zuckerberg][Facebook CEO][2014]
Virtual reality was once the dream of science fiction. But the internet was also once a dream, and so were computers and smartphones. The future is coming.
\end{fquote}

\begin{fquote}[Tim Cook][Apple CEO][2017]
I think AR is [...] big, it’s huge. I get excited because of the things that could be done that could improve a lot of lives.
\end{fquote}

\section{Einführung}
Viele der einflussreichsten Technologieunternehmen arbeiten an der \emph{Virtual} bzw.\ \emph{Augmented Reality (VR} bzw.\ \emph{AR)} über. Tim Cook ist überzeugt davon, dass die AR die nächste ``big idea'' nach dem Smartphone wird. \cite{theindependent2017}

Nicht nur Großkonzerne wie Apple, Facebook oder Samsung arbeiten intensiv in diesem Bereich. Ein Start-up-Unternehmen namens \emph{Magic Leap} entwickelt eine AR Brille und wird von Investoren in Billionenhöhe unterstützt. \cite{kelly2016untold} Laut einem Cover-Artikel der Zeitschrift \emph{Wire} ist die noch unter Verschluss gehaltene Technologie den Konkurrenzprodukten allen voraus. Das Release der Hardware ist Stand heute noch nicht bekannt, aber es lässt sich ein Trend erkennen, der die nächsten Jahre viele neue Möglichkeiten eröffnen wird. Diese Arbeit wird sich einer Möglichkeit davon widmen -- der Softwarevisualisierung.

Unter anderem sieht Tim Cook mehr Zukunft in der AR, da diese Technologie nicht wie die VR die wirkliche Welt ausschließt, sondern die Realität erweitert und Teil davon ist. Das ist der große Unterschied zwischen den beiden Technologien.

Obwohl der Begriff AR zunehmend in der Industrie und Literatur Verwendung findet, entbehrt er doch einer einheitlichen Definition. Es sollte zunächst Klarheit über die Begriffe VR, AR und auch Mixed Reality (MR) geschaffen werden.

\paragraph{VR} ist eine Umgebung, in der der Betrachter vollkommen von einer computergenerierten Welt umgeben ist, die oft die reale Welt imitiert, aber auch rein fiktiv sein kann. (vgl. \cite{milgram1995augmented})

\paragraph{AR} ist die Erweiterung der realen Welt durch computergenerierte Elemente, mit denen der Betrachter in Echtzeit interagieren kann.

\paragraph{MR} wird in der Industrie umgangssprachlich auch oft als Synonym von AR verwendet, da reale und virtuelle Realitäten "`gemixt"' werden. Jedoch wird in wissenschaftlichen Abhandlungen und deshalb auch in dieser Arbeit MR als Überbegriff von Realitäten verstanden, die

\section{Motivation dieser Arbeit}

\section{Zielsetzung}

\section{Aufbau der Arbeit}

\chapter{Das Konzept CodeLeaves}
\chapter{Datenmodell}
\chapter{Layout}
\chapter{Interaktionskonzept}
\chapter{Zusammenfassung und Ausblick}
