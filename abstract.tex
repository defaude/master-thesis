\chapter*{Kurzfassung}
\thispagestyle{empty}

Visualisierung spielt in vielen Bereichen der Wissenschaft eine wichtige Rolle. In der Informatik ist die Visualisierung von Software bei der Größe und Komplexität zeitgemäßer Software-Systeme eine hilfreiche Unterstützung für deren Entwicklung, Exploration und Analyse.

Mit dem zunehmend Einzug haltenden Medium der Augmented Reality bekommt die Softwarevisualisierung neue Möglichkeiten und wird mit 3D Modellen für den Nutzer greifbar.

CodeLeaves stellt ein neuartiges Konzept vor, mit dem eine Software mithilfe der Wald-Metapher zu dreidimensionalen Bäumen wird. Klassen werden zu Blättern und Pakete zu Ästen und ganzen Bäumen. Durch die vorhandene Baumstruktur in der Software ist mit CodeLeaves der geistige Transfer zwischen Realität und Abstraktion besonders gering.

Das Blätterdach des Waldes kann von einem sommerlichen Grün über Gelb bis hin zu einem herbstlichen Rot eine beliebige Metrik darstellen. Verbindungen in einer Software, seien es statische Abhängigkeiten oder dynamische Aufrufe, können übersichtlich auf die Baumstruktur aggregiert und interaktiv exploriert werden.

Dafür wird ein sprachunabhängiges Software-Meta-Modell entworfen, mit dem es möglich ist, Informationen der statischen und dynamischen Softwarevisualisierung darzustellen.

In einem Prototyp für die HoloLens, einer Augmented-Reality-Brille von Microsoft, wird die Generierung eines Waldes mit realistischen Beispieldaten erprobt und die dafür benötigten Layout-Algorithmen entwickelt. Dabei werden in der Natur vorkommende Verteilungsstrategien und hierarchisches Circle-Packing zur Platzierung der Elemente verwendet.

Die Interaktion mit dem Wald wird unter Einbeziehung von Herausforderungen in der Augmented Reality teils in dem Prototypen für die HoloLens implementiert und teils konzeptionell weiter ausgearbeitet.

\bigskip
\noindent \textbf{Schlagworte:} Konzept, 3D, Softwarevisualisierung, Wald, Metapher, Baum, Augmented Reality, Software-Meta-Modell, Statik, Qualitäts-Metriken, Abhängigkeiten, Dynamik, Aufrufe, Laufzeiten, 3D Layout-Algorithmen, Sonnenblumen-Algorithmus, Circle-Packing, Interaktion, Reaktive Programmierung
